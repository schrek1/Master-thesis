\documentclass[12pt, a4paper]{report}
\usepackage[utf8]{inputenc}
\usepackage[czech]{babel}
\usepackage{graphicx}
\usepackage{cite}
\usepackage{tabularx}
\usepackage{textgreek} % řecká písmena
\usepackage[bf,sf]{titlesec} % nastaveni formatovani kapitol
\usepackage[top=2.5cm, bottom=2.5cm, left=3cm, right=2cm]{geometry} % nastavi okraje stranek
\usepackage[hidelinks]{hyperref} % skryje url linky
\usepackage{indentfirst} % nastaveni odsazeni odstavce
\setlength{\parindent}{10mm} % nastavi velikost odstavce
\usepackage{color} % povolí barevné písmo

% nastaveni formatovani kodu start
\usepackage{listings}

\usepackage[dvipsnames]{xcolor}
\definecolor{dkgreen}{rgb}{0,0.6,0}
\definecolor{gray}{rgb}{0.5,0.5,0.5}
\definecolor{mauve}{rgb}{0.58,0,0.82}

\lstdefinelanguage{Kotlin}{ % nastaveni formatovani pro Kotlin
  comment=[l]{//},
  commentstyle={\color{gray}\ttfamily},
  emph={delegate, filter, first, firstOrNull, forEach, lazy, map, mapNotNull, println, return@},
  emphstyle={\color{OrangeRed}},
  identifierstyle=\color{black},
  keywords={abstract, actual, as, as?, break, by, class, companion, continue, data, do, dynamic, else, enum, expect, false, final, for, fun, get, if, import, in, interface, internal, is, null, object, override, package, private, public, return, set, super, suspend, this, throw, true, try, typealias, val, var, vararg, when, where, while},
  keywordstyle={\color{NavyBlue}\bfseries},
  morecomment=[s]{/*}{*/},
  morestring=[b]",
  morestring=[s]{"""*}{*"""},
  ndkeywords={@Deprecated, @JvmField, @JvmName, @JvmOverloads, @JvmStatic, @JvmSynthetic, Array, Byte, Double, Float, Int, Integer, Iterable, Long, Runnable, Short, String},
  ndkeywordstyle={\color{BurntOrange}\bfseries},
  sensitive=true,
  stringstyle={\color{ForestGreen}\ttfamily},
}

\lstset{ 
	frame=tb, % tb, none
	framexleftmargin=3em,
	xleftmargin=0em,
	xrightmargin=0em,
	language=Java,
	aboveskip=3mm,
	belowskip=3mm,
	showstringspaces=false,
	columns=flexible,
	basicstyle={\small\ttfamily},
	numbers=left, % nebo none
	numberstyle=\tiny\color{gray},
	numbersep=0em,    
	keywordstyle=\color{blue},
	commentstyle=\color{dkgreen},
	stringstyle=\color{mauve},
	breaklines=true,
	breakatwhitespace=true,
	tabsize=3,
	literate=~{ }1 % nahrazuje ~ (tildu) prázdnou mezerou
}% nastaveni formatovani kodu konec


% konstanty
\newcommand{\ConstAuthorName}{Bc. Ondřej Schrek}
\newcommand{\ConstWriteDate}{\today}

\begin{document} 
	\pagestyle{empty} % zákaz číslování stran
	
	\begin{titlepage} % titulní strana
		\centering
		\begin{center}
			\textsc{\LARGE Vysoká škola ekonomická v~Praze}\\[1.5cm]
			\textsc{\Large Fakulta informatiky a~statistiky}\\[0.5cm]
			\textsc{\large Katedra informačních technologií}\\[1.5cm]
			
			\textsc{Studijní program: Aplikovaná informatika}\\[0.5cm]
			\textsc{Obor: Informační systémy a~technologie}\\
			\vspace*{\fill}
			
			{\huge \bfseries{Prototyp aplikace pro ověření identity studenta u~písemné zkoušky }}\\[1.5cm]
			\textbf{DIPLOMOVÁ PRÁCE}
		\end{center}
	
		\vspace*{\fill}
		
		\begin{flushleft} % odsazení vlevo
			{\large
				\textbf{Student: \ConstAuthorName}\\
				\textbf{Vedoucí: Ing. Libor Gála, Ph.D.}
			}
		\end{flushleft}
	
		\vfill
		
		\begin{minipage}[b]{\textwidth}
			\begin{center}
				\large \textbf{2019}
			\end{center}
		\end{minipage}

	\end{titlepage} % konec titulní strana
	
	\clearpage\mbox{}\clearpage % volná strana
	
	% PROHLÁŠENÍ
	\vspace*{\fill}
	\section*{Prohlášení}
	\noindent Prohlašuji, že jsem diplomovou práci zpracoval samostatně
	a~že jsem uvedl všechny použité prameny a~literaturu, ze kterých jsem čerpal\\[1.5cm]
	\noindent V~Praze dne \ConstWriteDate \hfill \dotfill\\
	\begin{minipage}{\textwidth}
		\hfill \ConstAuthorName \hspace*{1cm}
	\end{minipage}
	% KONEC PROHLÁŠENÍ
	
	\clearpage\mbox{}\clearpage % volná strana
	
	\vspace*{\fill}
	% PODĚKOVÁNÍ
	\section*{Poděkování}
	\noindent Text poděkování.
	% KONEC PODĚKOVÁNÍ
	
	\clearpage\mbox{}\clearpage % volná strana
	
	% ABSRATKT
	\section*{Abstrakt}
	Doplnit text.
	
	\section*{Klíčová slova}
	Doplnit klíčová slova.
	% KONEC ABSRATKT
	
	\clearpage\mbox{}\clearpage % volná strana
	
	% ABSRATKT ANJ
	\section*{Abstract}
	Doplnit anj text.
	
	\section*{Keywords}
	Doplnit anj klíčová slova.
	% KONEC ABSRATKT ANJ
	
	\clearpage\mbox{}\clearpage % volná strana
	
	% OBSAH
	\tableofcontents % generování obsahu
	\renewcommand{\thepage}{\arabic{page}} % arabské číslování stránek
	\renewcommand{\baselinestretch}{1.5} % řádkování 1.5
	\addtocontents{toc}{\protect\thispagestyle{empty}} % obsah nebud číslovaný
	% KONEC OBSAH 
	
	\normalsize % default velikost písma
	
	% ÚVOD
	\chapter*{Úvod}
	\addcontentsline{toc}{chapter}{Úvod} % přidá kapitolu do obsahu
	\pagestyle{plain} % začnou se číslovat stránky
	\setcounter{page}{1} % pro úvod se nastaví číslo 1
	V dnešní době se téměř každá instituce nějakým způsobem potýká s problémem efektivního identifikování osob. 
	Vysoké školy jsou jednou z těchto institucí, kde se také potýkají s tímto problémem. Jelikož je zde potřeba často identifikovat větší skupiny studentů v co nejkratším čase.
	Příkladem je identifikace studentů u zkoušek, kdy zkoušející musí identifikovat každého studenta, 
	zda-li je zapsán na zkoušku a také jestli se jedná skutečně o osobu daného studenta.
	Na Vysoké škole ekonomické v Praze je potřeba řešení pro efektivnější a spolehlivější způsob identifikace. Řešení které bude reflektovat současné technické možnosti a 
	rozšiřovat systém, který je již na VŠE nastaven. Mezi další požadavky patří také to, aby řešení bylo možné snadno plošně nasadit při zachování nízkých nákladů.
	\\
	
	Cílem práce je vytvořit prototyp aplikace, který umožní identifikovat studenty u zkoušky.
	Tento prototyp bude realizován v podobě mobilní aplikace pro zařízení se systémem Android.
	Implementace bude provedena v jazyce Kotlin.
	Aplikace bude umožňovat načítání informací z identifikačních karet používaných studenty.
	Aplikace bude propojená s informačním systémem školy, ze kterého bude získávat informace o zkouškách a jejich účastnících.
	\\
	
	Součástí této práce je analýza současného stavu řešení systému identifikace a možností jeho zlepšení.
	Podle analýzou nalezených slabých míst a problémů bude navrhnut cílový stav systému identifikace.
	Poté budou zvoleny technologie pro implementaci a přístupy jakými lze dosáhnout cílového stavu systému.
	V souladu s vybraným přístupem bude navrhnuta integrace prototypu s informačním systémem InSIS.
	Dále se vydefinují úkoly k implementaci a vytvoří se samotný postup implementace.
	Podle navrhovaného postupu bude popsána implementace jednotlivých úkolů.
	V další části bude navrženo a popsáno testování prototypu. Poslední část práce nastíní možnosti dalšího rozvoje aplikace.
	
	% KONEC ÚVOD
	
	% KAPITOLA CHARAKTERISTIKA PROBLÉMU
	\chapter{Charakteristika problému}
		\section{Současný stav systému}
		V současné době se na VŠE používá k identifikaci studentů u zkoušek kontrola proti vytištěným seznamům přihlášených studentů.
		Zkoušející musí procházet seznam a vyhledávat studenty dle jejich školních identifikačních karet, nebo dokladů totožnosti, kterými se na začátku zkoušky prokazují.
		Při zkouškách kde je nahlášeno mnoho studentů je tato činnost velmi zdlouhavá a vyžaduje po zkoušejícím přípravu navíc. 
		Minimálně v podobě tisku aktuálních seznamů studentů těsně před zkouškou. Současný systém umožňuje identifikaci studenta pomocí načtení karty při vstupu do místnosti,
		avšak toto není dostatečně přesný způsob pro identifikaci konkrétního studenta z důvodu nemožnosti ztotožnit zda-li se skutečně jedná o danou osobu, 
		která se prokázala identifikační kartou.
		Tento způsob se na VŠE nově používá pro automatickou docházku na výuku, kde je tento způsob identifikace dostatečný. 
		Další nevýhodou je, že se tento systém dá využít pouze v učebnách, kde je přístup řízen pomocí karet. Na VŠE existují učebny, které tento přístup nemají 
		a v těchto učebnách by ani nebylo možné tuto metodu identifikace použít.
		\\
		
		Studentům jsou v současné době poskytovány dva typy bezkontaktní identifikačních karet, interní karta školy a mezinárodní karta ISIC.
		Obě karty jsou v rámci školního užívání rovnocenné. Obě tyto karty využívají technologii RFID-NFC.
		\section{Problémy stávajícího systému}
		Současný systém je neflexiblní a nabízí příležitosti pro jeho automatizaci. 
		Současné řešení neodpovídá technickým možnostem, které jsou již na VŠE zavedeny např. elektronické identifikační karty.
		Další nevýhodou je, že zkoušející musí zadávat prezenci do systému InSIS manuálně.
		\section{Souhrn problémů a~podmínek řešení}
		\section{Návrh řešení}
		\section{Cíl práce}
	% KONEC KAPITOLY CHARAKTERISTIKA PROBLÉMU
	
	% ZÁVĚR
	\chapter*{Závěr}
	\addcontentsline{toc}{chapter}{Závěr} % přidá závěr do obsahu
	Text závěru.
	% KONEC ZÁVĚR
	
	% CITACE
	\bibliographystyle{alpha} % typ číslování odstavce
	\bibliography{Zdroje}
	% KONEC CITACE

	\newpage

	\begin{lstlisting} % formátování jako zdrojový kód Java
		// Hello.java
		import javax.swing.JApplet;
		import java.awt.Graphics;

		public class Hello extends JApplet {
			public void paintComponent(Graphics g) {
				g.drawString("Hello, world!", 65, 95);
			}    
		}
	\end{lstlisting}
	
	\begin{lstlisting} % formátování jako zdrojový kód Java
	
	
	
	
	
		package com.example.admin.androidactivity;
	    import android.os.Bundle;
	    import android.support.annotation.Nullable;
	    import android.support.design.widget.FloatingActionButton;
	    import android.support.design.widget.Snackbar;
	    import android.support.v7.app.AppCompatActivity;
	    import android.support.v7.widget.Toolbar;
	    import android.view.View;
	    import android.view.Menu;
	    import android.view.MenuItem;
		
		public class ActivityExample extends AppCompatActivity {
		    @Override
		    protected void onStart() {
		        super.onStart();
		    }
		
		    @Override
		    protected void onCreate(@Nullable Bundle savedInstanceState) {
		        super.onCreate(savedInstanceState);
		    }
		
		    @Override
		    protected void onStop() {
		        super.onStop();
		    }
		
		    @Override
		    protected void onResume() {
		        super.onResume();
		    }
		
		    @Override
		    protected void onPause() {
		        super.onPause();
		    }
		
		    @Override
		    protected void onDestroy() {
		        super.onDestroy();
		    }
		}
	\end{lstlisting}
	
	\begin{lstlisting}[caption={Simple code listing.}, label={lst:example1}, language=Kotlin]
		/* Block comment */
		package hello
		import kotlin.collections.* // line comment
		
		/**
		 * Doc comment here for `SomeClass`
		 * @see Iterator#next()
		 */
		@Deprecated("Deprecated class")
		private class MyClass<out T : Iterable<T>>(var prop1 : Int) {
		    fun foo(nullable : String?, r : Runnable, f : () -> Int, 
		        fl : FunctionLike, dyn: dynamic) {
		        println("length\nis ${nullable?.length} \e")
		        val ints = java.util.ArrayList<Int?>(2)
		        ints[0] = 102 + f() + fl()
		        val myFun = { -> "" };
		        var ref = ints.size
		        ints.lastIndex + globalCounter
		        ints.forEach lit@ {
		            if (it == null) return@lit
		            println(it + ref)
		        }
		        dyn.dynamicCall()
		        dyn.dynamicProp = 5
		    }
		    
		    val test = """hello
		                  world
		                  kotlin"""
		    override fun hashCode(): Int {
		        return super.hashCode() * 31
		    }
		}
		fun Int?.bar() {
		    if (this != null) {
		        println(message = toString())
		    }
		    else {
		        println(this.toString())
		    }
		}
		var globalCounter : Int = 5
		    get = field
		abstract class Abstract {
		}
		object Obj
		enum class E { A, B }
		interface FunctionLike {
		    operator fun invoke() = 1
		}
	\end{lstlisting}
	
\end{document}